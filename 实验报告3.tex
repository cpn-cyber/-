\documentclass[a4paper, 12pt]{article}
\usepackage[UTF8]{ctex}
\usepackage{graphicx}
\usepackage{mathtools}
\usepackage{color}



\begin{document}
    
        \begin{figure}[htp]
            \centering
            \includegraphics[scale=1]{1.jpg}
        \end{figure}
        
        \begin{center}
            \kaishu\zihao{1} 实验报告三
            \end{center}
            \begin{center}
                \begin{tabular}{ll}
                    \kaishu\zihao{2} 课程: 系统开发工具基础\\
                    \kaishu\zihao{2} 姓名: 陈培诺\\
                    \kaishu\zihao{2} 学号: 23160001003\\
                    \kaishu\zihao{2} 时间: 2024年9月10日\\
                \end{tabular}
        \end{center}
    \pagenumbering{arabic}
    \tableofcontents
    \newpage
  
   
    \section{命令行环境}
   \subsection{实例1: session的创建 查看 重命名 关闭等命令}
        \noindent 创建名称为test的session:  tmux new -s test \\
        退出当前session: tmux detach\\
        查看session:  tmux ls\\
        进入session: tmux a -t test \\
        重命名为test1: tmux rename -t test test1 \\
        关闭session:  tmux kill-session -t test
        \begin{figure}[h!]
          \centering
          \includegraphics[width=1\textwidth]{2.jpg}
          \caption{实例1 }
        \end{figure}
        \newpage
        \subsection{实例2: window的创建 查看 重命名 关闭等命令}
        \noindent 创建名称为abc的window:  tmux new-window -n abc\\
        重命名为test1: tmux rename-window -t def hij \\
        关闭session:  tmux kill-window -t abc
        \begin{figure}[h!]
          \centering
          \includegraphics[width=1\textwidth]{3.jpg}
          \caption{实例2 }
        \end{figure}
        \newpage
        \subsection{实例3: 切割窗口}
        \noindent 水平切割窗口: tmux split-window
        竖直切割窗口: tmux split-window -h
        \begin{figure}[h!]
          \centering
          \includegraphics[width=1\textwidth]{4.jpg}
          \caption{实例3: 水平切割窗口 }
        \end{figure}
        \begin{figure}[h!]
          \centering
          \includegraphics[width=1\textwidth]{5.jpg}
          \caption{实例3: 竖直切割窗口 }
        \end{figure}
        \subsection{实例4: alias起别名}
        \noindent 创建vim的别名为v: alias v="vim"\\
        查看别名v的定义: alias v
        \begin{figure}[h!]
          \centering
          \includegraphics[width=1\textwidth]{6.jpg}
          \caption{实例4: 定义及查看别名 }
        \end{figure}
        \subsection{实例5: 生成ssh秘钥}
        \begin{figure}[h!]
          \centering
          \includegraphics[width=1\textwidth]{7.jpg}
          \caption{实例5: 创建ssh秘钥 }
        \end{figure}
        \newpage
        \section{Python基础}
        \subsection{实例6: 调用turtle库 绘制六边形}
        \begin{figure}[h!]
          \centering
          \includegraphics[width=1\textwidth]{8.jpg}
          \caption{实例6: 绘制六边形代码 }
        \end{figure}
        \begin{figure}[h!]
          \centering
          \includegraphics[width=1\textwidth]{9.jpg}
          \caption{实例6: 六边形效果图 }
        \end{figure}
        \newpage
        \subsection{实例7: 实现递归}
        阶乘的计算需要用到递归思想
        \begin{figure}[h!]
          \centering
          \includegraphics[width=1\textwidth]{10.jpg}
          \caption{实例7: 源代码 }
        \end{figure}
        \begin{figure}[h!]
          \centering
          \includegraphics[width=1\textwidth]{11.jpg}
          \caption{实例7: 效果图 }
        \end{figure}
        \newpage
        \subsection{实例8: 排序}
        \noindent 首先创建一个列表ls = [2, 5, 4, 1, 3]\\
        升序:\\
        ls.sort()\\
        print("升序排列后的列表:", ls)\\
        降序:\\
        ls.sort(reverse=True)\\
        print("降序排列后的列表:", ls)
        \begin{figure}[h!]
          \centering
          \includegraphics[width=1\textwidth]{12.jpg}
          \caption{实例8: 结果 }
        \end{figure}
        \subsection{实例9: 列表的比较}
        \noindent 列表之间进行比较,以相同的下标进行比较,如果值相同则比较下一组元素,如果值不同则直接出结果\\
        例如 ls = [2, 5, 4, 1, 3]
        ab = [6, 8, 9, 11, 2]\\
        由于2小于6 所以返回True
        \begin{figure}[h!]
          \centering
          \includegraphics[width=1\textwidth]{13.jpg}
          \caption{实例9: 比较 }
        \end{figure}
        \subsection{实例10: 元组与列表的转换}
        \noindent a=(1,2,3,4,5)\\
        b=list(a) 将元组转换成列表\\
        c=tuple(b) 将列表转换成元组\\
        print(b,c)
        \begin{figure}[h!]
          \centering
          \includegraphics[width=1\textwidth]{14.jpg}
          \caption{实例10: 元组和列表 }
        \end{figure}
        \subsection{实例11: 返回字典中的所有值}
        \noindent d=\{'a':1,'b':2,'c':5\}\\
        c=d.values()\\
        print(list(c))
        \begin{figure}[h!]
          \centering
          \includegraphics[width=1\textwidth]{15.jpg}
          \caption{实例11: 结果 }
        \end{figure}
        \subsection{实例12: while的循环}
        \noindent for循环在Python的语法跟c语言的语法不太一样\\
        例如要求1加到10\\
        sum=0\\
        for i in range(1,11)\\
        sum+=i\\
        print(sum)\\
        其中range括号里是左闭右开
        \newpage
        \subsection{实例13: 字符串大小写}
        \begin{figure}[h!]
          \centering
          \includegraphics[width=1\textwidth]{16.jpg}
          \caption{实例13: upper为大写 lower为小写 }
        \end{figure}
        \begin{figure}[h!]
          \centering
          \includegraphics[width=1\textwidth]{17.jpg}
          \caption{实例13: 结果}
        \end{figure}
        \subsection{实例14:转移字符}
        \noindent $\backslash$n是换行符\\
        $\backslash$'是单引号\\
        $\backslash$"是双引号\\
        $\backslash$$\backslash$是$\backslash$
        \subsection{实例15: 文件的读写}
        \noindent open("xxx.txt","w",encoding="UTF-8")\\
        第一个表示文件地址,第二个表示打开文件方式,w是只写的方式打开,r是表示以只读的方式打开,第三个表示文件的编码方式
        \subsection{实例16: 猜数游戏}
        调用random随机库
        \begin{figure}[h!]
          \centering
          \includegraphics[width=1\textwidth]{18.jpg}
          \caption{实例16}
        \end{figure}
       
        \subsection{实例17: 类和对象}
        定义一个student类 wo是对象
        \begin{figure}[h!]
          \centering
          \includegraphics[width=1\textwidth]{19.jpg}
          \caption{实例17}
        \end{figure}
        \subsection{实例18: 多行输出}
        三个引号即可实现多行输出
        \begin{figure}[h!]
          \centering
          \includegraphics[width=1\textwidth]{20.jpg}
          \caption{实例18}
        \end{figure}
        \begin{figure}[h!]
          \centering
          \includegraphics[width=1\textwidth]{21.jpg}
          \caption{实例18}
        \end{figure}
        \newpage
        \section{Python视觉应用}
        \subsection{实例19: 绘制图像点线}
        \begin{figure}[h!]
          \centering
          \includegraphics[width=1\textwidth]{22.jpg}
          \caption{实例19}
        \end{figure}
        \begin{figure}[h!]
          \centering
          \includegraphics[width=1\textwidth]{23.jpg}
          \caption{实例19}
        \end{figure}
        \newpage
        \newpage
        \subsection{实例20: 绘制图像直方图}
        \noindent figure()\\
        hist(im.flatten(),128)
        \begin{figure}[h!]
          \centering
          \includegraphics[width=1\textwidth]{24.jpg}
          \caption{实例20}
        \end{figure}
        \newpage
        \section{git仓库链接}
        https://github.com/cpn-cyber/-.git
        \section{学习心得}
        经过这次实验,我系统学习了命令行环境,python基础并了解了python视觉应用,通过以上二十个实例,也有助于我巩固对这些知识的掌握

        \par
        这节课只是初步学习这些知识,在今后我还得继续深入学习python,为我以后做相关开发奠定基础
      \end{document}
      \end{document}