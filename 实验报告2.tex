\documentclass[a4paper, 12pt]{article}
\usepackage[UTF8]{ctex}
\usepackage{graphicx}
\usepackage{mathtools}
\usepackage{color}



\begin{document}
    
        \begin{figure}[htp]
            \centering
            \includegraphics[scale=1]{1.jpg}
        \end{figure}
        
        \begin{center}
            \kaishu\zihao{1} 实验报告二
            \end{center}
            \begin{center}
                \begin{tabular}{ll}
                    \kaishu\zihao{2} 课程: 系统开发工具基础\\
                    \kaishu\zihao{2} 姓名: 陈培诺\\
                    \kaishu\zihao{2} 学号: 23160001003\\
                    \kaishu\zihao{2} 时间: 2024年8月30日\\
                \end{tabular}
        \end{center}
    \pagenumbering{arabic}
    \tableofcontents
    \newpage
  
   
    \section{Shell 工具和脚本}
   \subsection{实例1: find命令}
\noindent find命令的一般形式为:find pathname -options [-print -exec -ok ...]\\
find命令的参数:\\
pathname: find命令所查找的目录路径。例如用.来表示当前目录,用/来表示系统根目录\\
-print: find命令将匹配的文件输出到标准输出\\
-ok: 和-exec的作用相同,只不过以一种更为安全的模式来执行该参数所给出的shell命令,在执行每一个命令之前,都会给出提示,让用户来确定是否执行\\

\begin{figure}[h!]
    \centering
    \includegraphics[width=1\textwidth]{2.jpg}
    \caption{实例1}
  \end{figure}

  \subsection{实例2: man命令}
  通过man指令可以查看shell中的指令帮助、配置文件帮助和编程帮助等信息\\
  \begin{figure}[h!]
    \centering
    \includegraphics[width=1\textwidth]{3.jpg}
    \caption{实例2 man命令获取gedit的信息}
  \end{figure}
  \subsection{实例3: 脚本计算1-100奇数和}
  思路: 定义一个变量来保存奇数的和 (sum=0)
  找出1-100的奇数,保存到另一个变量里 (i=遍历出来的奇数)
  从1-100中找出奇数后,再相加,然后将和赋值给变量 (循环变量 for) 遍历完毕后,将sum的值打印出来
  \begin{figure}[h!]
    \centering
    \includegraphics[width=1\textwidth]{4.jpg}
    \caption{实例3 shell脚本内容}
  \end{figure}
  \begin{figure}[h!]
    \centering
    \includegraphics[width=1\textwidth]{5.jpg}
    \caption{实例3 shell脚本执行:结果为2500}
  \end{figure}

  \subsection{实例4: shell脚本判断所输整数是否为质数}
  思路: 让用户输入一个数,保存到一个变量里,如果能被其他数整除就不是质数
  如果输入的数是1或者2取模根据上面判断又不符合,所以先排除1和2
  \begin{figure}[h!]
    \centering
    \includegraphics[width=1\textwidth]{7.jpg}
    \caption{实例4 shell脚本内容}
  \end{figure}
  \begin{figure}[h!]
    \centering
    \includegraphics[width=1\textwidth]{6.jpg}
    \caption{实例4 shell脚本执行结果}
  \end{figure}

  \subsection{实例5: touch命令创建文本}
形式为touch xxx.后缀 如 touch abc.txt 创建一个文件名为abc的文本\\
touch abc.sh 创建一个名为abc的shell脚本文件


 \subsection{实例6: ls命令}
 ls命令可以查看指定目录下包含哪些文件
 \begin{figure}[h!]
    \centering
    \includegraphics[width=1\textwidth]{8.jpg}
    \caption{实例6 查看test1这一目录下有哪些文件}
  \end{figure}

  \subsection{实例7: pwd命令}
  pwd命令可以查看当前所处文件位置
  \begin{figure}[h!]
    \centering
    \includegraphics[width=1\textwidth]{9.jpg}
    \caption{实例7 pwd路径查询结果}
  \end{figure}
  \subsection{实例8: cat命令}
  cat命令可以显示文本内容
  \begin{figure}[h!]
    \centering
    \includegraphics[width=1\textwidth]{10.jpg}
    \caption{实例9 cat显示test1文本内容}
  \end{figure}
  \subsection{cp命令拷贝文件}
\noindent 复制文件:将一个或多个文件复制到指定的目标位置,
  例如: cp file1.txt file2.txt将file1.txt复制为file2.txt\\
复制目录:使用-r选项可以递归复制整个目录及其子目录,
例如: cp -r dir1 dir2将dir1目录及其所有子目录和文件复制到dir2目录

\subsection{实例10: shell脚本输出九九乘法表}
\begin{figure}[h!]
    \centering
    \includegraphics[width=1\textwidth]{11.jpg}
    \caption{实例10 shell脚本内容}
  \end{figure}
  \begin{figure}[h!]
    \centering
    \includegraphics[width=1\textwidth]{12.jpg}
    \caption{实例10 shell脚本执行结果}
  \end{figure}
\section{vim文本编辑器}
\subsection{实例11 vi/vim}
通过vi/vim +文件路径 即可进入编辑文件状态
\begin{figure}[h!]
    \centering
    \includegraphics[width=1\textwidth]{13.jpg}
    \caption{实例11 vi/vim进入编辑}
  \end{figure}
  \subsection{实例12: 插入模式}
  输入i即可在光标位置开始输入文本
  \begin{figure}[h!]
    \centering
    \includegraphics[width=1\textwidth]{14.jpg}
    \caption{实例12 输入HELLO WORLD}
  \end{figure}
  \subsection{实例13: wq指令}
  使用wq指令即可实现 保存文件并退出
  \subsection{实例14: 删除字符}
  键盘输入x即可删除当前光标的字符
  \begin{figure}[h!]
    \centering
    \includegraphics[width=1\textwidth]{15.jpg}
    \caption{实例14 删除"虽"这一字符}
  \end{figure}
  \subsection{实例15: 撤销上一次操作}
  键盘输入u即可撤销上一次操作
  \begin{figure}[h!]
    \centering
    \includegraphics[width=1\textwidth]{16.jpg}
    \caption{实例15 撤销删除"虽"的操作}
    \end{figure}
    \subsection{实例16: 进入底线命令模式}
    键盘输入"-"即可进入底线命令模式
    \subsection{实例17: 从编辑模式切换命令模式}
    若输入了i进入编辑模式,想切换回命令模式,输入ESC即可
    \subsection{实例18: 统计文件字符数}
    键盘输入wc -m +文件名 即可统计文件字符数
    \begin{figure}[h!]
        \centering
        \includegraphics[width=1\textwidth]{17.jpg}
        \caption{实例18 统计abc.txt文件 字符数为 348}
      \end{figure}
    \subsection{实例19: 粘贴}
    在命令模式下输入P即可实现粘贴
    \begin{figure}[h!]
        \centering
        \includegraphics[width=1\textwidth]{18.jpg}
        \caption{实例19 将原来的内容粘贴了一次}
      \end{figure}
    \subsection{实例20: 四个退出指令的区别}
    \noindent :q  当vim进入文件没有对文件内容做任何操作可以按"q"退出 \\
    :q!  当vim进入文件对文件内容有操作但不想保存退出 \\
    :wq  正常保存退出 \\
    :wq! 强行保存退出,只针对与root用户或文件所有人生效 
    \section{git仓库链接}
    https://github.com/cpn-cyber/-.git
    \section{学习心得}
    经过这次实验,我系统学习了shell工具及其脚本,vim编辑器并了解了数据整理方法,通过以上二十个实例,也有助于我巩固对这些知识的掌握
    \par
    在shell中我认为我对for循环这块知识还不够熟练,还需继续练习其编程题。而在vim编辑中我需要要牢牢记住那些重要的命令
    \par
    这节课只是初步学习这些知识,在今后我还得继续深入学习shell工具以及脚本,为我以后做相关开发奠定基础
  \end{document}