\documentclass[a4paper, 12pt]{article}
\usepackage[UTF8]{ctex}
\usepackage{graphicx}
\usepackage{mathtools}
\usepackage{color}



\begin{document}
    \begin{figure}[h]
        \centering
        \includegraphics[scale=2]{0.jpg}
  
    \end{figure}
    \begin{center}
    \heiti\zihao{0} 实验报告
    \end{center}
    \begin{center}
        \begin{tabular}{ll}
         \heiti\zihao{2} 姓名: 陈培诺\\
         \heiti\zihao{2}学号: 23160001003\\
         \heiti\zihao{2}时间: 2024年8月26日
    \end{tabular}
\end{center}


\newpage

\section{学习收获}
\subsection{版本控制git}
\begin{enumerate}
    \item {\large git config}
    \begin{itemize}
    \item{配置修改用户名及邮箱 git config --global user.name "xxx"或git config -- global user.email "xxx"}
    \end{itemize}
    \item{\large git status}
    \begin{itemize}
      \item 查看相关文件的状态(位于工作区,暂存区或者版本区)
    \end{itemize}
    \item{\large git commit}
    \begin{itemize}
      \item 将处于暂存区的文件提交至版本区中,git commit -m "提交的说明" 
    \end{itemize}
  \item{\large git reset --soft}
    \begin{itemize}
      \item 把该版本号提交的内容从暂存区位置回滚到工作区位置 git reset --soft "版本号" 
    \end{itemize}
    \item{\large git reset --hard}
    \begin{itemize}
      \item 把该版本号提交的内容从版本区位置回滚到工作区位置 git reset --hard "版本号" 
    \end{itemize}
    \item{\large git reset --mixed}
    \begin{itemize}
      \item 把该版本号提交的内容从版本区位置回滚到暂存区位置 git reset --mixed "版本号" 
    \end{itemize}
    \item{\large git add}
    \begin{itemize}
      \item 将工作区的文件提交至暂存区中,git add "文件名"\\git add .将之前的所有未提交到到暂存区的文件提交至暂存区
    \end{itemize}
    \newpage
    \item {\large git init} 
    \begin{itemize}
      \item 初始化一个Git仓库 
    \end{itemize}
    \item {\large git clone url}
    \begin{itemize}
      \item 克隆远程版本库\quad其中url为git仓库地址
    \end{itemize}
    \item {\large git log}
    \begin{itemize}
      \item 查看历史提交记录
    \end{itemize}
    \item {\large git branch}
    \begin{itemize}
      \item 显示所有本地分支
    \end{itemize}
    \item {\large git branch "name"}
    \begin{itemize}
      \item 创建新分支,并命名
    \end{itemize}
    \item {\large git branch -d"name"}
    \begin{itemize}
      \item 删除指定分支
    \end{itemize}
    \item {\large git checkout"branch name"}
    \begin{itemize}
      \item 切换到指定分支
    \end{itemize}
    \item {\large git merge "branch name"}
    \begin{itemize}
      \item 合并指定分支至当前分支
    \end{itemize}
    \item {\large git rebase "branch name"}
    \begin{itemize}
      \item 衍合指定分支到当前分支
    \end{itemize}
    \item {\large git remote "xxx"}
    \begin{itemize}
      \item 查看远程版本库信息
    \end{itemize}
    \newpage
    \item {\large git tag}
    \begin{itemize}
      \item 列出所有本地标签
    \end{itemize}
    \item {\large git tag "tagname"}
    \begin{itemize}
      \item 基于最新提交创建标签
    \end{itemize}
    \item {\large git tag -d "tagname"}
    \begin{itemize}
      \item 删除指定标签
    \end{itemize}
\end{enumerate}
\subsection{latex排版}
\begin{enumerate}
    \item {\large 表格设计}
    \begin{itemize}
        \item 
        \begin{tabular}{|c|c|c|}
        \hline \textbf{a}&\textbf{b}&\textbf{c}\\
        \hline 1 & 2 & 3\\
        \hline 4 & 5 & 6\\
        \hline 7 & 8 & 9\\
        \hline
        \end{tabular}
      \end{itemize}
      \item{\large 字体控制}
      \begin{itemize}
        \item 字体颜色
        \begin{center}
            \item {\color{red}abcdefg}
            \item {\heiti 陈培诺(黑体字)}
          \end{center}
        \item 字体背景颜色
        \begin{center}
            \item \colorbox{green}{\color{red}abcdef}
          \end{center}
        \newpage
        \item 字体标注下划线
        \begin{center}
            \underline{abcdefg}
        \end{center}
        \item 加粗
        \begin{center}
            \textbf{陈培诺}
        \end{center}
     
      \end{itemize}
      \item {\large 数学公式}
      \begin{itemize}
        \item eg1:
        \begin{equation}
        \alpha^2+\beta^2=\gamma^2
      \end{equation}
      \item eg2:
      \begin{equation}
        F(x)=
        \begin{cases}
        10&,\text{if $x<0$}\\
        100&,\text{if $x=0$}\\
        x+10&,\text{if $x>0$}
        \end{cases}
      \end{equation}
      \end{itemize}
      \item{\large 文献引用}
      \begin{itemize}
        \item I am a boy\cite{1}
      \end{itemize}
      \begin{thebibliography}{99}
        \bibitem{1} {http://www.latexstudio.net/}
      \end{thebibliography}
    \end{enumerate}
    \subsection{创建github仓库}
    https://github.com/cpn-cyber/-.git
    \section{学习心得}
    通过这次课程的学习,首先我深深体会到git管理版本的优势,不用像以前那样傻傻的复制原文件,而且学会这项技能也有助于未来在公司管理项目,再来是latex,以前的我只会用word写文章,排版耗时且不美观,如今学会了更加先进的工具 latex,对我未来完成科研论文以及毕业论文奠定了基础,
    当然我现在只是初步入门这些工具,在学习的过程中也遇到了许多问题如latex排版命令行的陌生,字体大小控制不熟练等等问题,在今后我会深入学习git和latex相关的知识,不断提高我对相关命令行的熟悉程度。
\end{document}